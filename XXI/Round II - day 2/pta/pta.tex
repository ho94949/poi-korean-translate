\begin{problem}{아기새}
	{standard input}{standard output}
	{3 seconds}{128 megabytes}{}
	
	지구이웨 국립공원에는 $n$개의 나무가 일렬로 자라고 있다. 첫번째 나무의 꼭대기에는 마지막 나무의 꼭대기로 날고싶어하는 아기새가 있다. 아기새는 매우 작기 때문에 마지막 나무까지 한번에 날아갈 힘이 없을 수도 있다. $i$번째 나무에 앉아있는 아기새는 한번 날기 시작해서 $i+1$, $i+2$, $\cdots$, $i+k$번째 나무까지 날 수 있고, 그 후로 가려면 쉬어야 한다.
	
	그리고 날아 올라가는 것은 날아 내려가는 것보다 힘든 일이다. 높이가, 시작한 나무의 높이보다 같거나 높은 곳으로 날아가면 피로가 1만큼 쌓인다. 그렇지 않은 경우에는 피로가 쌓이지 않는다.
	
	아기새의 목표는 피로를 최대한 적게 쌓으면서 마지막 나무로 가는 것이다. 알다시피, 새들은 사회적인 동물이기 때문에 아기새는 똑같은 목표를 가지고 첫째 나무에서 마지막 나무로 가고싶어하는 서로 다른 체력 $k$를 가진 아기새들이 있다. 아기새들을 도와주자!
	
	
	\InputFile
	입력의 첫째 줄에는 지구이웨 국립공원에 있는 나무의 수 $n$이 주어진다. ($2 \le n \le 1,000,000$) 둘째 줄에는 $n$개의 정수 $d_1, d_2, \cdots, d_n$이 공백 하나로 구분되어 주어진다. ($1 \le d_i \le 10^9$) $d_i$는 $i$번째 나무의 높이이다.
	
	셋째 줄에는 마지막 나무로 가고 싶어하는 새의 수를 나타내는 $q$가 주어진다. ($1 \le q \le 25$) 다음 $q$개의 줄 중 $i$번째 줄에는 $i$번째 새의 힘을 나타내는 $k_i$가 주어진다. ($1 \le k \le n-1$) 즉, $i$번째 새는 쉬지 않고 연속으로 $k_i - 1$개의 나무를 지나칠 수 있다.
	
	
	\OutputFile
	
	프로그램은 $q$개의 줄을 출력해야 한다. $i$번째 줄에는 $i$번째 새가 마지막으로 가는데 까지 얻은 피로를 의미한다.
	
	\SubtaskWithCost{1}{33}
	\begin{itemize}
		\item $n \le 1,000$
	\end{itemize}
	
	\SubtaskWithCost{2}{44}
	\begin{itemize}
		\item $n \le 100,000$
	\end{itemize}
	
	\SubtaskWithCost{3}{23}
	추가 제한 조건이 없다.
	
	
	
	\Examples
	
	\begin{example}
	\exmp{
	9
	4 6 3 6 3 7 2 6 5
	2
	2
	5
	}{%
	2
	1
	
	}%
\end{example}

\Note

첫번째 새는 1, 3, 5, 7, 8, 9번 나무에서 쉬어간다. 3번째에서 5번째 나무로 갈 때, 7번째에서 8번째 나무로 갈 때 피로를 얻는다.

\end{problem}

