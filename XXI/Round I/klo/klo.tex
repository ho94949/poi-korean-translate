\begin{problem}{블럭}
	{standard input}{standard output}
	{3 seconds}{64 megabytes}{}
	
	어제 범수와 친구들은 유치원에서 다양한 색의 블럭을 만들고 노느라 하루를 다 보냈다. 처음에 그들은 모형 빌딩을 만들었지만 곧 질려버렸다. 그래서 그들은 블럭을 일렬로 배치하기로 했다. 단조로워 보이는 것을 막기 위해 같은 색의 두 블럭을 옆에 놓지 않으려고 노력했다. 긴 시간끝에 그들은 이 조건에 맞춰 블럭을 놓는데에 성공했고, 부모님들은 다시 범수와 친구들을 집으로 데려갔다.
	 
	오늘 범수는 유치원에 일찍 왔다. 그는 어제 쌓은 블럭이 계속 남아있다는 것에 만족했다. 하지만 그는 그 블럭 위로 넘어지고 말았다. 범수는 그 블럭들을 색별로 정리하고 다시 원래대로 복구하고 싶어 했다. 그가 기억한 것은 가장 왼쪽 끝과 오른쪽 끝의 블럭 색 뿐이다.
	 
	범수에게 두 인접한 블럭이 같은 색을 가지지 않고, 그가 기억하던 왼쪽 끝과 오른쪽 끝의 색을 유지하면서 어떻게 블럭들을 배치해야 하는지 알려주자. 범수가 기억을 잘못하거나, 넘어지면서 블럭들을 잃어버려서 새로 배치하는것이 불가능 할수도 있다.
	
	
	
	\InputFile
	
	첫째 줄에는 세 정수 블럭 색의 가짓수 $k$, 가장 왼쪽 끝 블럭의 색 $p$, 가장 오른쪽 끝 블럭의 색 $q$가 공백 하나로 구분되어 주어진다. ($1 \le k \le 1,000,000$, $1 \le p, q \le k$)
	둘째 줄에는 $k$개의 정수 $i_1, i_2, \cdots, i_k$가 공백 하나로 구분되어 주어진다. $i_j$는 색 $j$의 블럭을 범수가 정확히 $i_j$개 가지고 있다는 것을 의미한다. 총 블럭의 갯수는 백만개를 넘지 않는다. 즉, $n=i_1 + i_2 + \cdots + i_k \le 1,000,000$ 이다.
	
	
	\OutputFile
	
	첫째 줄에 $n$개의 수를 공백 하나로 구분하여 출력하여라. 이 수는 가능한 블럭 배치의 왼쪽 부터 오른쪽 까지의 색을 나타낸다. 만약 그런 배치가 없다면, 0을 출력하여라.
	
	답이 여러개일 경우, 아무거나 출력하여도 좋다.
	
	\Examples
		
	\begin{example}
	\exmp{
	3 3 1
	2 3 3
	}{%
	3 2 1 3 2 3 2 1
	}%
	\exmp{
	3 3 1
	2 4 2
	}{%
	0
	}%
	\end{example}
	
	\Note
	
	첫번째 예제에서, 올바른 다른 정렬은 ``3 1 2 3 2 3 2 1" 이다. 두번째 예제에서, 범수가 어딘가 실수를 해서 원래 조건을 만족하지 못했다.
\end{problem}

