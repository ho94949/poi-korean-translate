\begin{problem}{택배}
	{standard input}{standard output}
	{3 seconds}{64 megabytes}{}
	
	범수는 컴퓨터게임을 파는 ZIG회사에서 일한다. 이 회사는 ZIG회사에서 판매되는 게임들을 공급하는 다른 게임회사와 같이 일한다. 범수는 ZIG회사와 택배회사와의 협력도 조사하고 있다. 그는 각 게임 패키지를 공급한 택배회사가 어느 회사인지에 대한 기록을 가지고 있다. 범수는 어떠한 회사도 불공평한 기회를 가지지 않았으면 한다.
	
	만약 어떤 택배회사가 특정한 기간에 절반 초과의 패키지를 보냈다고 하면, 그 회사가 그 기간을 독점했다고 말한다. 범수는 어떤 기간에 특정 택배회사가 독점을 한 적이 있는지 알아보고 싶다.
	 
	범수를 도와주자! 독점하는 택배회사가 있는지 없는지 결정하는 프로그램을 작성하여라.
	
	
	
	\InputFile
	
	첫째 줄에는 ZIG회사가 배송한 게임 패키지의 수를 나타내는 $n$과, 게임 패키지를 배송한 날의 수를 나타내는 $m$이 공백 하나로 구분되어 들어온다. ($1 \le n, m \le 500,000$)
	택배 회사들은 1이상 최대 $n$이하의 수로 표현된다.
	둘째 줄에는 $n$개의 정수 $p_1, p_2, \cdots, p_n$이 공백 하나로 구분되어 들어온다. ($1 \le p_i \le n$) $p_i$는 $i$번째로 배송된 게임 패키지의 번호를 나타낸다. 
	다음 $m$개의 줄에는 쿼리가 한 줄에 하나씩 공백 하나로 구분된 두 수 $a$, $b$로 주어진다. ($1 \le a \le b \le n$) 이 쿼리는 $a$번째 패키지부터 $b$번째 패키지 까지 배송하는 기간에 독점하는 회사가 있는지 없는지 확인해야 한다. (이 기간은 $a$번째 패키지와 $b$번째 패키지를 포함한다.)
	

	
	\OutputFile
	
	각 쿼리당 한 줄에 하나씩 답을 출력해야 한다. (총 $m$줄을 출력해야 한다.) 각 줄에는 정수 하나가 있어야 한다. 쿼리에 해당하는 구간에 독점하는 회사가 있다면 그 회사의 번호를, 없다면 0을 출력 해야 한다.
	
	
	\SubtaskWithCost{1}{30}
	\begin{itemize}
		\item $n, m \le 5,000$
	\end{itemize}
	
	\SubtaskWithCost{2}{35}
	\begin{itemize}
		\item $n, m \le 50,000$
	\end{itemize}
	
	\SubtaskWithCost{3}{35}
	
	추가 제한조건이 없다.
	
	\Examples
		
	\begin{example}
	\exmp{
	7 5
	1 1 3 2 3 4 3
	1 3
	1 4
	3 7
	1 7
	6 6
	}{%
	1
	0
	3
	0
	4
	}%
	\end{example}
	
	
	
	
	
\end{problem}

