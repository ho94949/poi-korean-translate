\begin{problem}{온도}
	{standard input}{standard output}
	{1 second}{128 megabytes}{}
	
	

지구이웨 국립연구소에서는 매일매일 온도를 측정한다. 측정은 전부 자동이고, 결과는 바로 프린터로 출력된다. 유감스럽게도, 프린터에 잉크가 다 떨어졌지만 최근 지구이웨 국립대학교가 자료를 요청하기 전까지는 알 지 못했다.

다행히도 지구이웨 국립연구소의 성실한 인턴 범수가 지구이웨 국립연구소의 북쪽과 남쪽에 있는 알코올 온도계에서 매일 매일 온도를 기록했다. 알코올 온도계는 부정확했지만 수십년간 쌓인 기록으로 미루어 볼 때, 남쪽에 있는 온도계는 실제 기온보다 낮지 않고, 북쪽에 있는 온도계는 실제 기온보다 높지 않은 온도를 표시한다는 것을 알 수 있다. 그래서 각 날짜의 정확한 기온은 알 수 없지만, 범위는 알 수 있었다.


운이 좋게도, 지구이웨 국립대학교는 정확한 온도를 요구하지는 않았다. 그들은 온도가 떨어지지 않는 가장 긴 연속된 기간을 알고 싶었다. (그 전날 보다 온도가 낮지 않는 것을 의미한다) 사실, 지구이웨 국립연구소의 소장은 지구이웨 국립대학교의 연구원들이 이 기간이 길수록 연구 결과가 잘 나왔다고 결론 지을 것이라는 사실을 알고 있다. 지구이웨 국립연구소의 소장은 범수의 기록과 모순되지 않는 온도가 떨어지지 않는 가장 긴 연속된 기간을 알고싶어 한다. 하지만 범수는 아직 인턴이기 때문에 이 일을 하지는 못한다. 숙련된 연구원인 당신은 이 문제를 푸는 프로그램을 작성해야 한다.


	\InputFile
	
	첫째 줄에는 범수가 온도를 기록한 날의 수를 의미하는 정수 $n$이 주어진다. ($1 \le n \le 1,000,000$) 다음 $n$개의 줄에 $i$번째 줄에는, 기록을 시작한지 $i$번째 날의 북쪽 온도계의 온도(최솟값)와 남쪽 온도계의 온도(최댓값)를 의미하는 두 정수 $x$, $y$가 공백 하나로 구분되어 주어진다. ($-10^9 \le x \le y \le 10^9$) 지구이웨는 전 세계에서 미국과 함께 섭씨를 쓰지 않는 두 나라 중의 하나이다. 
	
	
	\OutputFile
	
	첫째 줄에, 범수의 기록과 모순되지 않는 온도가 떨어지지 않는 가장 긴 연속된 기간의 길이를 출력하여라.
	
	\SubtaskWithCost{1}{50}
	\begin{itemize}
		\item $-50 \le x \le y \le 50$
	\end{itemize}
	
	\SubtaskWithCost{2}{50}
	
	추가 제한 조건이 없다.
	
	\Examples
		
	\begin{example}
	\exmp{
		6
		6 10
		1 5
		4 8
		2 5
		6 8
		3 5
		
	}{%
	4
	}%
	\end{example}
	
	
        
\end{problem}

