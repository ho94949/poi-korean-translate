\begin{problem}{벽돌 매장}
	{standard input}{standard output}
	{1 second}{64 megabytes}{}
	
	범수는 벽돌 매장을 운영한다. 이번 시즌의 베스트셀러는 라미네이트 바닥재이다. 불행하게도 창고에 바닥재가 없어 고객이 주문을 할 수 없는 경우도 있다. 고객이 없어지는것을 막기 위해서 범수는 고객이 주문을 할 수 없는 경우를 최소화 하려고 한다.
	
	이를 위해 범수는 $n$일에 대한 작업 일정을 마련했다 그는 바닥 생산자와의 계약을 분석 했으며 수열 $a_1, a_2, \cdots, a_n$을 얻었다. $a_i$는 $i$번째 날 아침에 $a_i$개의 바닥재가 매장으로 도달한다는 뜻이다.
 
	범수는 또 도매업자의 주문을 분석해서 수열 $b_1, b_2, \cdots, b_n$을 얻었다. 이것은 $i$번째 날 정오에 고객이 $b_i$개의 바닥재를 주문한다는 뜻이다. 만약 범수가 주문을 받기로 결정한다면, 그는 $b_i$만큼의 바닥재를 주어야 한다. 만약에 바닥재가 충분하지 않다면 그는 주문을 거절하여야 하고, 만약 충분하다면 범수가 주문을 받을지 말지 결정할 수 있다.
	
	이 정보에 기반하여 범수는 어떤 주문을 받아야 그가 거절하는 주문의 수가 (고의든 아니든) 최소가 되는지 궁금해 졌다. 우리는 첫째날 샙겨에 매장이 비어있다고 가정한다.

	

	\InputFile
	
	첫째 줄에는 정수 $n$이 주어진다. ($1 \le n \le 250,000$). 둘째 줄에는 수열 $a_1 , a_2, \cdots, a_n$이, 셋째 줄에는 수열 $b_1, b_2, \cdots, b_n$이 주어진다. ($0 \le a_i, b_i \le 10^9$) 숫자들은 공백 하나로 구분되어 있다.
 
	
	\OutputFile
	
	첫째 줄에는 범수가 수용 가능한 최대 주문의 갯수를 출력한다. 둘째 줄에는 증가하는 순서로 $k$개의 숫자가 공백 하나로 구분되어 출력되어야 한다. 이 숫자는 해당하는 날에 오는 주문을 받을 수 있다는 것을 의미한다. 만약에 어떤 주문도 받을 수 없다면 둘째줄은 비어 있어야 한다. 답이 여러개 있을 경우 아무거나 출력하여도 좋다. 
	
	\SubtaskWithCost{1}{50}
	\begin{itemize}
		\item $n \le 1,000$
	\end{itemize}
	
	\SubtaskWithCost{2}{50}
	
	추가 제한조건이 없다.
	
	\Examples
		
	\begin{example}
	\exmp{
6
2 2 1 2 1 0
1 2 2 3 4 4
	}{%
3
1 2 4
	}%
	\end{example}
	
	\Note
	
\end{problem}

